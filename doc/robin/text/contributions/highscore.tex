\subsection{Highscore \& Datenbankverbindung}

Das Highscore System und die Datenbankverbindung ermöglichen es einen Highscore sowohl zu speichern, als auch die $n$ größten Highscores anzuzeigen.
Dabei ist $n$ in \texttt{Globals} frei einstellbar\footnote{Die Anzahl ist standardmäßig $n=10$.}.
Die Anzeige der Highscores erscheint zum Ende eines Spiels, wenn entweder weniger als $n$ Highscores existieren oder der erreichte Score größer als der kleinste ist.
Erfüllt der erreichte Score diese Kriterien, so wird der Spieler aufgefordert seinen frei wählbaren Namen anzugeben\footnote{Die Länge des Namens ist auch in \texttt{Globals} standardmäßig auf $10$ Zeichen limitiert.}.
Die Angabe des Namens ist jedoch optional, das Spiels kann auch ohne Eintragung neu gestartet werden.
Die Highscore Daten können entweder in eine Online-Datenbank oder offline in eine \gls{csv} Datei gespeichert werden.
Standardmäßig ist die Offline \gls{csv} Version eingestellt.
Das kann wieder in \texttt{Globals} verändert werden.
Die Implementation dieser gesamten Funktionalität konzentriert sich auf die Klassen \texttt{Dbc} (kurz für \textit{DataBase Connection}), \texttt{HighscoreSystem} und \texttt{HighscoreEntry}.
Außerdem hat das System Einzugsbereiche im \texttt{GameLoop}.
Zu einem wird die Anzeige daraus gestartet und erhält den aktuellen Score.
Zum anderen kann die Datenbank, egal ob online oder offline, durch eine Passworteingabe im \textit{Credits} Bildschirm zurückgesetzt bzw. gelöscht werden.
Für die Online-Datenbank wird das Administrator Passwort der Datenbank benötigt.Für die Offline-Datenbank ist es das \texttt{DefaultPassword} in \texttt{Globals}.

Die Online Datenbank ist eine \gls{mysql} Datenbank, welche mit \texttt{preparedStatements} arbeitet.
Die Offline Datenbank benutzt \textit{Apache}'s \texttt{commons-csv} Bibliothek um die Daten in eine \gls{csv} Datei zu schreiben.
Die Implementation befindet sich in \texttt{Dbc}.
Die Klasse \texttt{Dbc} ist final und hat keinen internen Zustand.
Somit werden zwar mehr Ressourcenaufrufe benötigt, jedoch wird sichergestellt, dass die angezeigten Daten stets so aktuell wie möglich sein können.

Die Klasse \texttt{HighscoreEntry} stellt einen einzelnen Highscore, ein Datenpaar aus Name und Score dar.
Diese werden im \texttt{HighscoreSystem} als Arraylist gespeichert.
Dort werden diese Daten und auch das Namenseingabefeld und die Überschrift graphisch angezeigt.
Die Texteingabe und Validierung in das Namensfeld wird auch darin gehandhabt.
Die \texttt{Dbc} wird folgend davon aufgerufen um entweder einen neuen Highscore zu schreiben, alle geschriebenen auszulesen oder um alle zu löschen.
