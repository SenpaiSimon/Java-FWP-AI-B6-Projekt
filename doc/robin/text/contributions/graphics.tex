\subsection{Graphik \& Animationen}

Alle Bilder und Animationen werden durch Sprites implementiert.
Diese sind im \texttt{sprites} Unterordner anzutreffen.
Sie wurden durch \textit{Adobe Photoshop}, \textit{After Effects} und \textit{Media Encoder} erstellt.
Bilder können durch den \texttt{ImageLoader} geladen werden.
Diese greift auf Dateien zu, daher stand die Implementation sehr in Verbindung mit dem Build-System.
Es musste sichergestellt werden, dass die zu ladenden Bilder auch in der \gls{jar} zugänglich sind.

Der \texttt{ImageLoader} läd Sprites so, dass diese durch die implizite Skalierung bei der Anwendung der Sprites auf ein Graphikobjekt ohne Anti-Aliasing skaliert werden.
Dadurch ersteht der Effekt, dass alle Sprites, egal wie groß, eine annehmbare Größe annehmen und dabei auch pixeliert werden.
Das hilft sofern, dass eigene Pixel-Art tatsächlich auf den Dimensionen der Pixel gezeichnet werden kann und zum Export nicht skaliert werden muss.
Beispielsweise können die Spieler Sprites im Format von $16 \times 16$ Pixel gezeichnet und gespeichert werden, die Hintergründe können aber wesentlich größer sein.
Im Spiel nehmen beide jedoch eine vergleichbare Skalierung an.

Animationen zwischen mehreren Sprites können im Spieler, den Projektielen, den Gegner und im Hauptmenü gefunden werden.
Die Animationen bestehen aus einem einfachen Timer, welcher nach definierter Zeit die Sprite wechselt.
Der Spieler verfügt über zwei verschiedene Animationszustände: Ein \textit{Idle} und ein \textit{Running} Zustand.
Je nach der momentanen Geschwindigkeit des Spielers werden die stationäre \textit{Idle} Animation oder die bewegende \textit{Running} Animation benutzt.
